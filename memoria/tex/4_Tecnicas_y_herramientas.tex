\capitulo{4}{Técnicas y herramientas}

Esta parte de la memoria tiene como objetivo presentar las técnicas metodológicas y las herramientas de desarrollo que se han utilizado para llevar a cabo el proyecto. Si se han estudiado diferentes alternativas de metodologías, herramientas, bibliotecas se puede hacer un resumen de los aspectos más destacados de cada alternativa, incluyendo comparativas entre las distintas opciones y una justificación de las elecciones realizadas. 
No se pretende que este apartado se convierta en un capítulo de un libro dedicado a cada una de las alternativas, sino comentar los aspectos más destacados de cada opción, con un repaso somero a los fundamentos esenciales y referencias bibliográficas para que el lector pueda ampliar su conocimiento sobre el tema.


\section{Herramientas de control de versiones}

\subsection{Github}

\subsection{Github aplicacion escritorio}

\section{Herramientas de gestión de proyectos}

\subsection{ZenHub}

\section{Metodologías}

\subsection{\textit{Sprints}}

\section{Patrones de diseño}

\section{Herramientas de evaluación de código}

\subsection{SonarQube}

\section{Herramientas de documentación}

\subsection{\LaTeX}

\subsection{Overleaf}

\subsection{Draw.io - Microsoft Visio}

\section{Entornos de desarrollo integrado (IDE)}

\subsection{Visual Studio Code}
Usaremos esta interfaz de desarrollo por sus grandes ventajas como puede ser su integración nativa con git, el soporte multiplataforma y multilenguaje, la alta personalización con gran cantidad de extensiones, además de ser gratuito y de código abierto.

\section{Lenguajes de programación}

\subsection{Python}
Se ha escogido python como lenguaje en el desarrollo del modelo porque es uno de los lenguajes más utilizados en el campo de la Inteligencia Artificial. Además de tener una sintaxis clara y simple, existe una amplia comunidad de desarrolladores, y bibliotecas especializadas en IA. Asimismo, es muy flexible y fácil de usar, lo que permite una fácil integración con otros lenguajes.

\subsection{CSharp ¿?}

\subsection{SQL}

\section{Librerías}


