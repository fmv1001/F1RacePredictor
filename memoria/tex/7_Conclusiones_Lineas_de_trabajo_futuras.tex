\capitulo{7}{Conclusiones y Líneas de trabajo futuras}

\section{Conclusiones}

Procedo a mostrar las conclusiones que se derivan del desarrollo del presente proyecto:

\begin{itemize}
\tightlist
\item
  El objetivo general del proyecto se ha cumplido con éxito. Se ha desarrollado un modelo de predicción de resultados de Fórmula 1, el cual puede utilizarse de una manera muy simple gracias a la interfaz de usuario por cualquier persona.
\item
  También se ha completado el ajuste de los modelos con éxito consiguiendo mejoras notables en alguno de ellos.
\item
  Por otra parte he comprobado que aunque la selección manual de características puede ser un buen enfoque, la selección mediante algoritmos como la eliminación recursiva de características son más eficientes computacionalmente en la mayor parte de las ocasiones.
\item
  Gracias a la realización del proyecto he profundizado en el conocimiento sobre inteligencia artificial, modelos predictivos y machine learning.
\item
  Este proyecto ha abarcado una buena parte de los conocimientos que he obtenido durante mi estancia en la universidad. Pero no sólo he hecho uso de ellos, si no que también he estudiado y adquirido nuevos conocimientos requeridos para la realización del proyecto. Entre ellos se encuentran: Webscrapping, Scikit-learn, \LaTeX, pyqt, etc.
\item
  He aprendido a hacer búsquedas bibliográficas rápidas y efectivas como resultado de los requerimientos de investigación del proyecto.
\item
  El período de desarrollo de este proyecto ha supuesto el uso de numerosas tecnologías y herramientas. Todas ellas han contribuido a elevar su calidad. Sin embargo, algunas de ellas han requerido una sobrecarga de trabajo considerable. A pesar de esto, la información aprendida será muy útil para futuros emprendimientos.

\end{itemize}

\section{Líneas de trabajo futuras}

A continuación, discutiremos cada extensión o mejora potencial que sería aplicable al trabajo actual.

\subsection{Aumentar la dimensionalidad de los datos}
Un punto de mejora en este proyecto sería lograr mayor cantidad de datos de clasificación, recopilar datos de los pit stops o de las vueltas rápidas, adelantamientos, etc.

\subsection{Añadir funciones a la aplicación}
Se podría añadir un apartado en la aplicación para añadir datos al conjunto y que se vuelvan a entrenar los modelos automáticamente, y si los datos mejoran, adoptar esos nuevos modelos.

\subsection{Interpretación de resultados}
Podrían desarrollarse métodos para comprender y justificar las opciones del modelo a medida que interpreta las decisiones. El usuario final puede beneficiarse de esto al sentirse más seguros de las predicciones y al poder comprender las variables que afectan a los resultados.

\subsection{Aplicación en dominios particulares}
Una buena aplicación de este proyecto podría ser su uso en dominios particulares como pueden ser análisis deportivo, simulaciones o videojuegos.
