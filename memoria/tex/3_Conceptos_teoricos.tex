\capitulo{3}{Conceptos teóricos}

Para comprender el marco teórico del desarrollo de este proyecto, es fundamental contar con un conocimiento previo de los conceptos en los que se basa.


\section{Fórmula 1} 
La \href{https://www.formula1.com/}{Fórmula 1} es una competición automovilística que dió inicio en 1950. En la actualidad es considerada la más prestigiosa del mundo y también es la serie deportiva anual con mayor popularidad. Este campeonato está regulado por la la FIA (Federación Internacional del Automóvil) y actualmente está compuesta por 10 equipos con dos coches cada uno. Dicho certámen se lleva a cabo desde marzo hasta noviembre, con un total de 23 carreras disputadas en 20 países distribuidos en cuatro continentes \cite{f1:f1}.

\section{Modelo predictivo}

Un modelo predictivo es un modelo de datos, basado en estadísticas inferenciales, que se utiliza para predecir resultados. Consiste en un conjunto de herramientas y técnicas estadísticas que sirven para pronosticar y predecir el comportamiento ante un evento. 

Aplicado al campo de la inteligencia artificial, podríamos decir que un modelo de predicción no es más que un modelo de caja negra para hacer predicciones \cite{art:predictmodel}, lo que significa que no se requiere un conocimiento detallado de su funcionamiento interno para utilizarlo. Los modelos predictivos utilizan técnicas basadas en el aprendizaje automático para hacer predicciones basadas en datos históricos o patrones identificados en los datos de entrenamiento. Dichos modelos aprenden automáticamente de los datos de entrenamiento y generan predicciones basadas en su capacidad para identificar patrones ocultos en los datos de entrada

\section{IA - Inteligencia Artificial}

La Inteligencia Artificial o IA se podría definir como la capacidad de las máquinas para tomar decisiones tal como lo hace el ser humano. Para ello se vale de diferentes algoritmos y en el aprendizaje sobre una gran cantidad de datos. Algunos sostienen que la inteligencia radica en las propiedades de los procesos internos de pensamiento y razonamiento, mientras que otros la definen en términos de comportamiento inteligente, una caracterización externa \cite{aimodaproach}.

\section{Python}

Python \cite{python} es un lenguaje de programación orientado a objetos, interpretado e interactivo que posee características como la inclusión de módulos, excepciones, tipado dinámico, tipos de datos de alto nivel y clases. Debido a su gran cantidad de librerías y programadores, es muy popular y ampliamente utilizado en diferentes campos de la programación, lo que permite realizar diversos propósitos. Además, se le considera un lenguaje de alto nivel que ofrece una sintaxis clara y sencilla de comprender. \\
Fue creado a principios de la década de 1990 por Guido van Rossum en Stichting Mathematisch Centrum (CWI) en los Países Bajos como sucesor de un idioma llamado ABC \cite{pythonhistory}. \\
Además, este lenguaje de programación es muy popular y ampliamente utilizado en el campo de la inteligencia artificial, ofreciendo una serie de librerías especializadas como pueden ser TensorFlow, Keras, PyTorch, y scikit-learn, que facilitan la implementación de algoritmos de aprendizaje automático y procesamiento de datos. \\
\imagen{most_used_languages}{lenguajes de programación mas usados desde 2010 a 2022 \cite{pythonuse}. Basado en el porcentaje de preguntas en \textit{Stack Overflow}, uno de los portales de preguntas y respuestas sobre programación más importantes de todo el mundo.}{}
\section{Referencias}

Las referencias se incluyen en el texto usando cite \cite{wiki:latex}. Para citar webs, artículos o libros \cite{wiki:latex}.


\section{Imágenes}

Se pueden incluir imágenes con los comandos standard de \LaTeX, pero esta plantilla dispone de comandos propios como por ejemplo el siguiente:

\imagen{escudoInfor}{Autómata para una expresión vacía}{.5}



\section{Listas de items}

Existen tres posibilidades:

\begin{itemize}
	\item primer item.
	\item segundo item.
\end{itemize}

\begin{enumerate}
	\item primer item.
	\item segundo item.
\end{enumerate}

\begin{description}
	\item[Primer item] más información sobre el primer item.
	\item[Segundo item] más información sobre el segundo item.
\end{description}
	
\begin{itemize}
\item 
\end{itemize}

\section{Tablas}

Igualmente se pueden usar los comandos específicos de \LaTeX o bien usar alguno de los comandos de la plantilla.

\tablaSmall{Herramientas y tecnologías utilizadas en cada parte del proyecto}{l c c c c}{herramientasportipodeuso}
{ \multicolumn{1}{l}{Herramientas} & App AngularJS & API REST & BD & Memoria \\}{ 
HTML5 & X & & &\\
CSS3 & X & & &\\
BOOTSTRAP & X & & &\\
JavaScript & X & & &\\
AngularJS & X & & &\\
Bower & X & & &\\
PHP & & X & &\\
Karma + Jasmine & X & & &\\
Slim framework & & X & &\\
Idiorm & & X & &\\
Composer & & X & &\\
JSON & X & X & &\\
PhpStorm & X & X & &\\
MySQL & & & X &\\
PhpMyAdmin & & & X &\\
Git + BitBucket & X & X & X & X\\
Mik\TeX{} & & & & X\\
\TeX{}Maker & & & & X\\
Astah & & & & X\\
Balsamiq Mockups & X & & &\\
VersionOne & X & X & X & X\\
} 
