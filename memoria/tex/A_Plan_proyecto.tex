\apendice{Plan de Proyecto Software}

\section{Introducción}

En el presente anexo se describirá la planificación llevada a cabo en el proyecto. La planificación es una etapa crucial en la cual se estiman los costos del proyecto, tanto en términos de trabajo, tiempo y recursos financieros. Analizar la planificación resulta sumamente beneficioso, puesto que permite determinar la viabilidad del proyecto, el tiempo necesario para su culminación, los recursos requeridos y su cronograma de empleo, entre otros aspectos relevantes. Para ello, se ha divido la planificación en dos fases fundamentales: la planificación temporal y el estudio de viabilidad.

\section{Planificación temporal}

Se va a seguir una planificación por \textit{sprints}, la cual se explica a continuación.

\subsection{Sprint 0: Planificación y Objetivos del Proyecto (1 semana)}

Planificar el desarrollo en sprints del proyecto.
La planificación del desarrollo del proyecto comienza con la definición de los objetivos del proyecto y los resultados esperados. Además en este punto vamos a dividir el trabajo en tareas mas pequeñas.

Establecer los objetivos principales del proyecto.
Tras su estudio se van a establecer como objetivos principales del proyecto comparar varios algoritmos para encontrar el más óptimo y desarrollar este para maximizar su desempeño.

Escoger los entornos de desarrollo, de gestión y de comunicación.
Como entorno de desarrollo vamos a utilizar Visual Sudio Code por la versatilidad que nos brinda, podremos desarrollar tanto la aplicación como el modelo en esta herramienta.


\subsection{Sprint 1: Comprensión del problema y recolección de datos (1 semana)} 

\begin{itemize}
\item
Comprender el dominio de la F1.
\item
Identificar y recopilar los datos necesarios para construir el modelo predictivo.
\item
Familiarizarse con las fuentes de datos disponibles.
\end{itemize}


\subsection{Sprint 2: Preparación de los datos (2 semanas)}

Descargar y limpiar los datos relevantes.

Realizar un análisis exploratorio de los datos (EDA) para identificar patrones y tendencias.

Preprocesar los datos para poder ser utilizados en la construcción del modelo.

\subsection{Sprint 3: Selección del modelo (2 semanas)}

Investigar y seleccionar los algoritmos de aprendizaje automático adecuados para el problema de predicción.

Evaluar y comparar los modelos candidatos utilizando validación cruzada.
Estudiar el tema e investigar los algoritmos utilizados en esta materia.

\subsection{Sprint 4: Ajuste del modelo (3 semanas)}

Ajustar los hiperparámetros del modelo seleccionado.

Optimizar el rendimiento del modelo mediante técnicas como la regularización y la selección de características.

Realizar pruebas de validación cruzada y ajustar el modelo si es necesario.

\subsection{Sprint 5: Evaluación del modelo (2 semanas)}
Evaluar el rendimiento del modelo en datos de prueba.

Utilizar técnicas de validación de modelo para asegurar que el modelo no está sobreajustando a los datos de entrenamiento.

Documentar el rendimiento del modelo y presentar los resultados.

\subsection{Sprint 6: Implementación y despliegue en una aplicación de interacción (2 semanas)}

Diseñar la arquitectura de la aplicación.

Configurar el entorno de desarrollo para la aplicación.

Desarrollar la lógica de la aplicación para la interacción con el modelo.
Desarrollar la interfaz de usuario de la aplicación.

Realizar pruebas exhaustivas de la aplicación y asegurarse de que funciona correctamente con datos de prueba y producción.

Recopilar comentarios de los usuarios y realizar mejoras en la interfaz de usuario.

Corregir cualquier error que se haya encontrado durante la implementación y despliegue de la aplicación.

\section{Estudio de viabilidad}

\subsection{Viabilidad económica}

\subsection{Viabilidad legal}


