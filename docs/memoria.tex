\documentclass[a4paper,12pt,twoside]{memoir}

% Castellano
\usepackage[spanish,es-tabla]{babel}
\selectlanguage{spanish}
\usepackage[utf8]{inputenc}
\usepackage[T1]{fontenc}
\usepackage{lmodern} % Scalable font
\usepackage{microtype}
\usepackage{placeins}
\usepackage{amssymb} % Símbolo de verificación

\RequirePackage{booktabs}
\usepackage{adjustbox}
\RequirePackage[table]{xcolor}
\RequirePackage{xtab}
\RequirePackage{multirow}

% Links
\PassOptionsToPackage{hyphens}{url}\usepackage[colorlinks]{hyperref}
\hypersetup{
	allcolors = {red}
}

% Ecuaciones
\usepackage{amsmath}

% Rutas de fichero / paquete
\newcommand{\ruta}[1]{{\sffamily #1}}

% Párrafos
\nonzeroparskip

% Huérfanas y viudas
\widowpenalty100000
\clubpenalty100000

% Imágenes

% Comando para insertar una imagen en un lugar concreto.
% Los parámetros son:
% 1 --> Ruta absoluta/relativa de la figura
% 2 --> Texto a pie de figura
% 3 --> Tamaño en tanto por uno relativo al ancho de página
\usepackage{graphicx}
\newcommand{\imagen}[3]{
	\begin{figure}[!h]
		\centering
		\includegraphics[width=#3\textwidth]{#1}
		\caption{#2}\label{fig:#1}
	\end{figure}
	\FloatBarrier
}

% Comando para insertar una imagen sin posición.
% Los parámetros son:
% 1 --> Ruta absoluta/relativa de la figura
% 2 --> Texto a pie de figura
% 3 --> Tamaño en tanto por uno relativo al ancho de página
\newcommand{\imagenflotante}[3]{
	\begin{figure}
		\centering
		\includegraphics[width=#3\textwidth]{#1}
		\caption{#2}\label{fig:#1}
	\end{figure}
}

% El comando \figura nos permite insertar figuras comodamente, y utilizando
% siempre el mismo formato. Los parametros son:
% 1 --> Porcentaje del ancho de página que ocupará la figura (de 0 a 1)
% 2 --> Fichero de la imagen
% 3 --> Texto a pie de imagen
% 4 --> Etiqueta (label) para referencias
% 5 --> Opciones que queramos pasarle al \includegraphics
% 6 --> Opciones de posicionamiento a pasarle a \begin{figure}
\newcommand{\figuraConPosicion}[6]{%
  \setlength{\anchoFloat}{#1\textwidth}%
  \addtolength{\anchoFloat}{-4\fboxsep}%
  \setlength{\anchoFigura}{\anchoFloat}%
  \begin{figure}[#6]
    \begin{center}%
      \Ovalbox{%
        \begin{minipage}{\anchoFloat}%
          \begin{center}%
            \includegraphics[width=\anchoFigura,#5]{#2}%
            \caption{#3}%
            \label{#4}%
          \end{center}%
        \end{minipage}
      }%
    \end{center}%
  \end{figure}%
}

%
% Comando para incluir imágenes en formato apaisado (sin marco).
\newcommand{\figuraApaisadaSinMarco}[5]{%
  \begin{figure}%
    \begin{center}%
    \includegraphics[angle=90,height=#1\textheight,#5]{#2}%
    \caption{#3}%
    \label{#4}%
    \end{center}%
  \end{figure}%
}
% Para las tablas
\newcommand{\otoprule}{\midrule [\heavyrulewidth]}
%
% Nuevo comando para tablas pequeñas (menos de una página).
\newcommand{\tablaSmall}[5]{%
 \begin{table}
  \begin{center}
   \rowcolors {2}{gray!35}{}
   \begin{tabular}{#2}
    \toprule
    #4
    \otoprule
    #5
    \bottomrule
   \end{tabular}
   \caption{#1}
   \label{tabla:#3}
  \end{center}
 \end{table}
}

%
% Nuevo comando para tablas pequeñas (menos de una página).
\newcommand{\tablaSmallSinColores}[5]{%
 \begin{table}[H]
  \begin{center}
   \begin{tabular}{#2}
    \toprule
    #4
    \otoprule
    #5
    \bottomrule
   \end{tabular}
   \caption{#1}
   \label{tabla:#3}
  \end{center}
 \end{table}
}

\newcommand{\tablaApaisadaSmall}[5]{%
\begin{landscape}
  \begin{table}
   \begin{center}
    \rowcolors {2}{gray!35}{}
    \begin{tabular}{#2}
     \toprule
     #4
     \otoprule
     #5
     \bottomrule
    \end{tabular}
    \caption{#1}
    \label{tabla:#3}
   \end{center}
  \end{table}
\end{landscape}
}

%
% Nuevo comando para tablas grandes con cabecera y filas alternas coloreadas en gris.
\newcommand{\tabla}[6]{%
  \begin{center}
    \tablefirsthead{
      \toprule
      #5
      \otoprule
    }
    \tablehead{
      \multicolumn{#3}{l}{\small\sl continúa desde la página anterior}\\
      \toprule
      #5
      \otoprule
    }
    \tabletail{
      \hline
      \multicolumn{#3}{r}{\small\sl continúa en la página siguiente}\\
    }
    \tablelasttail{
      \hline
    }
    \bottomcaption{#1}
    \rowcolors {2}{gray!35}{}
    \begin{xtabular}{#2}
      #6
      \bottomrule
    \end{xtabular}
    \label{tabla:#4}
  \end{center}
}

%
% Nuevo comando para tablas grandes con cabecera.
\newcommand{\tablaSinColores}[6]{%
  \begin{center}
    \tablefirsthead{
      \toprule
      #5
      \otoprule
    }
    \tablehead{
      \multicolumn{#3}{l}{\small\sl continúa desde la página anterior}\\
      \toprule
      #5
      \otoprule
    }
    \tabletail{
      \hline
      \multicolumn{#3}{r}{\small\sl continúa en la página siguiente}\\
    }
    \tablelasttail{
      \hline
    }
    \bottomcaption{#1}
    \begin{xtabular}{#2}
      #6
      \bottomrule
    \end{xtabular}
    \label{tabla:#4}
  \end{center}
}

%
% Nuevo comando para tablas grandes sin cabecera.
\newcommand{\tablaSinCabecera}[5]{%
  \begin{center}
    \tablefirsthead{
      \toprule
    }
    \tablehead{
      \multicolumn{#3}{l}{\small\sl continúa desde la página anterior}\\
      \hline
    }
    \tabletail{
      \hline
      \multicolumn{#3}{r}{\small\sl continúa en la página siguiente}\\
    }
    \tablelasttail{
      \hline
    }
    \bottomcaption{#1}
  \begin{xtabular}{#2}
    #5
   \bottomrule
  \end{xtabular}
  \label{tabla:#4}
  \end{center}
}



\definecolor{cgoLight}{HTML}{EEEEEE}
\definecolor{cgoExtralight}{HTML}{FFFFFF}

%
% Nuevo comando para tablas grandes sin cabecera.
\newcommand{\tablaSinCabeceraConBandas}[5]{%
  \begin{center}
    \tablefirsthead{
      \toprule
    }
    \tablehead{
      \multicolumn{#3}{l}{\small\sl continúa desde la página anterior}\\
      \hline
    }
    \tabletail{
      \hline
      \multicolumn{#3}{r}{\small\sl continúa en la página siguiente}\\
    }
    \tablelasttail{
      \hline
    }
    \bottomcaption{#1}
    \rowcolors[]{1}{cgoExtralight}{cgoLight}

  \begin{xtabular}{#2}
    #5
   \bottomrule
  \end{xtabular}
  \label{tabla:#4}
  \end{center}
}

\graphicspath{ {./img/} }

% Capítulos
\chapterstyle{bianchi}
\newcommand{\capitulo}[2]{
	\setcounter{chapter}{#1}
	\setcounter{section}{0}
	\setcounter{figure}{0}
	\setcounter{table}{0}
	\chapter*{#2}
	\addcontentsline{toc}{chapter}{#2}
	\markboth{#2}{#2}
}

% Apéndices
\renewcommand{\appendixname}{Apéndice}
\renewcommand*\cftappendixname{\appendixname}

\newcommand{\apendice}[1]{
	%\renewcommand{\thechapter}{A}
	\chapter{#1}
}

\renewcommand*\cftappendixname{\appendixname\ }

% Formato de portada
\makeatletter
\usepackage{xcolor}
\newcommand{\tutor}[1]{\def\@tutor{#1}}
\newcommand{\course}[1]{\def\@course{#1}}
\definecolor{cpardoBox}{HTML}{E6E6FF}
\def\maketitle{
  \null
  \thispagestyle{empty}
  % Cabecera ----------------
\noindent\includegraphics[width=\textwidth]{cabecera}\vspace{1cm}%
  \vfill
  % Título proyecto y escudo informática ----------------
  \colorbox{cpardoBox}{%
    \begin{minipage}{.8\textwidth}
      \vspace{.5cm}\Large
      \begin{center}
      \textbf{TFM del Máster Universitario en Ingeniería Informática}\vspace{.6cm}\\
      \textbf{\LARGE\@title{}}
      \end{center}
      \vspace{.2cm}
    \end{minipage}

  }%
  \hfill\begin{minipage}{.20\textwidth}
    \includegraphics[width=\textwidth]{escudoInfor}
  \end{minipage}
  \vfill
  % Datos de alumno, curso y tutores ------------------
  \begin{center}%
  {%
    \noindent\LARGE
    Presentado por \@author{}\\ 
    en Universidad de Burgos --- \@date{}\\
    Tutor: \@tutor{}\\
  }%
  \end{center}%
  \null
  \cleardoublepage
  }
\makeatother

\newcommand{\nombre}{Francisco Martín Vargas} %%% cambio de comando
\newcommand{\nombretutor}{Daniel Urda Muñoz} 
\newcommand{\titulo}{Modelo predictivo de resultados de carreras de Fórmula 1} 

% Datos de portada
\title{\titulo}
\author{\nombre}
\tutor{\nombretutor}
\date{\today}

\begin{document}

\maketitle


\newpage\null\thispagestyle{empty}\newpage


%%%%%%%%%%%%%%%%%%%%%%%%%%%%%%%%%%%%%%%%%%%%%%%%%%%%%%%%%%%%%%%%%%%%%%%%%%%%%%%%%%%%%%%%
\thispagestyle{empty}


\noindent\includegraphics[width=\textwidth]{cabecera}\vspace{1cm}

\noindent D. \nombretutor, profesor del departamento de Digitalización, área de Ciencia de la Computación e Inteligencia Artificial.

\noindent Expone:

\noindent Que el alumno D. \nombre, con DNI 71704736M, ha realizado el Trabajo final de Máster Universitario en Ingeniería Informática titulado \titulo. 

\noindent Y que dicho trabajo ha sido realizado por el alumno bajo la dirección del que suscribe, en virtud de lo cual se autoriza su presentación y defensa.

\begin{center} %\large
En Burgos, {\large \today}
\end{center}

\vfill\vfill\vfill

% Author and supervisor
\begin{minipage}{0.45\textwidth}
\begin{flushleft} %\large
Vº. Bº. del Tutor:\\[2cm]
D. Daniel Urda Muñoz
\end{flushleft}
\end{minipage}
\hfill

\hfill

\vfill

% para casos con solo un tutor comentar lo anterior
% y descomentar lo siguiente
%Vº. Bº. del Tutor:\\[2cm]
%D. nombre tutor


\newpage\null\thispagestyle{empty}\newpage




\frontmatter

% Abstract en castellano
\renewcommand*\abstractname{Resumen}
\begin{abstract}
La \textit{Inteligencia Artificial} o \textit{IA} ha cambiado significativamente nuestras vidas y muchas facetas de nuestra sociedad. Gracias a la eficacia y la precisión que demuestran, han conseguido mejorar la comodidad de nuestras actividades diarias mediante la automatización de tareas repetitivas. Además, se han creado nuevas posibilidades en la industria del transporte, la salud, la educación, e incluso en la competición del motor, impulsando su desarrollo y elevando su nivel. Su gran capacidad de analizar y procesar información en aspectos sobre el rendimiento del vehículo, o el comportamiento del conductor y las estrategias de carrera han proporcionado a los equipos de competición gran cantidad de diferentes perspectivas para abordar la toma de decisiones de los equipos.

Un modelo de predicción de resultados resume a la perfección la fusión de la inteligencia artificial y la competición automovilística, poniendo al alcance de todo el mundo las tecnologías que se usan en lo más alto del mundo del motor. Con ese propósito se desarrollará un modelo para pronosticar los resultados de las carreras.

El presente documento muestra el empleo de técnicas de \textit{aprendizaje automático} para realizar predicciones sobre los resultados en la competición automovilística \textit{Fórmula 1}. Se pretender llevar a cabo este proyecto a través del lenguaje de programación \textit{Python}, así como proporcionar una aplicación desde la cual podremos gestionar dichas predicciones. Para llevarlo a cabo, se han utilizado el entorno de desarrollo \textit{Visual Studio Code}.

El resultado de este modelo de predicción se encuentra en el siguiente repositorio de GitHub: \href{https://github.com/fmv1001/F1RacePredictor}{F1RacePredictor}.
\end{abstract}

\renewcommand*\abstractname{Descriptores}
\begin{abstract}
Inteligencia Artificial, Modelo predictivo, Aprendizaje automático, Fórmula 1, Python.
\end{abstract}

\clearpage

% Abstract en inglés
\renewcommand*\abstractname{Abstract}
\begin{abstract}
Artificial Intelligence (AI) has significantly changed our lives and many facets of our society. Thanks to the efficiency and precision they demonstrate, they have managed to improve the convenience of our daily activities by automating repetitive tasks. In addition, new possibilities have been created in the transport industry, healthcare, education, and even in motor racing, driving their development and raising their level. Its strong ability to analyse and process information on aspects of vehicle performance, or driver behaviour and race strategies has provided racing teams with a wealth of different perspectives to approach team decision making.

A performance prediction model perfectly encapsulates the fusion of artificial intelligence and motor racing, making the technologies used at the top of the motor racing world available to everyone. To this end, a model for predicting race results will be developed.

This document shows the use of machine learning techniques to make predictions about results in Formula 1 motor racing. The aim is to carry out this project using the Python programming language, as well as to provide an application from which we will be able to manage these predictions. To carry it out, the development environment \textit{Visual Studio Code} has been used.

The result of this prediction model can be found in the following GitHub repository: \href{https://github.com/fmv1001/F1RacePredictor}{F1RacePredictor}.
\end{abstract}

\renewcommand*\abstractname{Keywords}
\begin{abstract}
Artificial Intelligence, Predictive Modeling, Machine Learning, Formula 1, Python.
\end{abstract}

\clearpage

% Indices
\tableofcontents

\clearpage

\listoffigures

\clearpage

\listoftables
\clearpage

\mainmatter
\capitulo{1}{Introducción}

Al igual que muchos otros aspectos de nuestras vidas, el mundo de la Fórmula 1 también se ha visto significativamente alterado por la inteligencia artificial (IA). Las IA se han abierto camino para alcanzar su máximo potencial en este emocionante deporte, donde la innovación, la velocidad y la estrategia son cruciales.

La Fórmula 1 es un evento automovilístico de primer nivel que combina ingeniería de vanguardia, capacidad de conducción extrema y tácticas de equipo sofisticadas. Cada milisegundo, cada elección y cada pequeño detalle importan en este entorno intensamente competitivo. Una décima de segundo puede decidir una victoria o una derrota.

El uso de la inteligencia artificial en la Fórmula 1 ha creado nuevas posibilidades para mejorar la eficiencia de los equipos y los pilotos. La IA puede revelar patrones ocultos y ofrecer información importante en la toma de decisiones estratégicas. Todo esto gracias al análisis de grandes conjuntos de datos obtenidos durante las carreras, como pueden ser los datos de telemetría o las condiciones climáticas.

Uno de los aspectos más emocionantes de la inteligencia artificial en la Fórmula 1 es la predicción del rendimiento y los resultados. Los modelos de aprendizaje automático pueden observar datos de todas las carreras, evaluar el rendimiento del piloto y del equipo, y señalar los factores que tienen el mayor impacto en el resultado de una carrera. Gracias a ello los equipos toman decisiones inteligentes con el objetivo de maximizar su rendimiento durante el gran premio, desde los ajustes en la configuración del coche hasta las estrategias de parada en boxes.

Las IA también han contribuido significativamente al diseño y la simulación de lo coches de Fórmula 1. Para encontrar la configuración que maximiza la velocidad con la estabilidad del coche, los algoritmos de optimización pueden examinar gran cantidad de combinaciones diferentes de ajustes mecánicos y configuraciones en la aerodinámica.

Finalmente, la inteligencia artificial ha logrado revolucionar la Fórmula 1, brindando importantes herramientas para mejorar su desempeño y la capacidad en la toma de decisiones. Ha cambiado permanentemente los deportes de élite, desde la predicción de resultados hasta la optimización del diseño de automóviles. Se deben tener en cuenta los potenciales usos de la inteligencia artificial y los modelos de aprendizaje automático en la Fórmula 1 y cómo se pueden aplicar a la vida diaria.

\section{Estructura de la memoria}

La memoria está organizada de la siguiente manera:

\begin{itemize}
\item
  \textbf{Introducción:} Describe el proyecto y da una visión general. También se puede acceder a la organización de la memoria, sus anexos y sus materiales adjuntos.
\item
  \textbf{Objetivos del proyecto:} objetivos que se tratará de cumplir al terminar el proyecto. 
\item
  \textbf{Conceptos teóricos:} breve explicación de las principales ideas teóricas necesarias para comprender el desarrollo del proyecto.
\item
  \textbf{Técnicas y herramientas:} colección de herramientas y técnicas metodológicas que se utilizaron para desarrollar el proyecto.
\item
  \textbf{Aspectos relevantes del desarrollo del proyecto:} presentación de los aspectos más importantes del proyecto.
\item
  \textbf{Trabajos relacionados:} presentación y comparación con algunos trabajos relacionados.
\item
  \textbf{Conclusiones y Líneas de trabajo futuras:} conclusiones extraídas de la finalización del proyecto, así como sugerencias para posibles avances en el desarrollo del proyecto en el futuro.
\end{itemize}

\section{Estructura de los anexos}

Los anexos se estructuran de la siguiente manera:

\begin{itemize}
\item
  \textbf{Plan del proyecto Software:} desarrollo de un estudio de viabilidad del proyecto y de la planificación temporal.
\item
  \textbf{Especificación de Requisitos:} requisitos resultantes de los objetivos del proyecto.
\item
  \textbf{Especificación de diseño:} descripción de la arquitectura del sistema y los diagramas resultantes.
\item
  \textbf{Documentación técnica de programación:} descripción de las herramientas (entornos de desarrollo, lenguajes, etc.) necesarias para trabajar en el proyecto.
\item
  \textbf{Documentación de usuario:} guía en la que se expone como se debe usar el producto final, con vistas al usuario final.
\end{itemize}

\subsection{Enlaces de los materiales del proyecto}

\begin{itemize}
\item
	\href{https://github.com/fmv1001/F1RacePredictor}{Repositorio del proyecto}.
\item
	\href{}{Vídeo demostración} (enlace a YouTube).
\end{itemize}
\capitulo{2}{Objetivos del proyecto}

El objetivo principal de este proyecto consiste en desarrollar un modelo predictivo para carreras de Fórmula 1. Por otro lado, se van a comparar diversos algoritmos del campo de la inteligencia artificial con el fin de identificar el más adecuado y optimizar al máximo su desempeño. Además, se pretende crear una aplicación que permita la interacción con el modelo final, ofreciendo una herramienta sencilla y accesible para los usuarios finales.

\section{Objetivos generales}\label{objetivos-generales}

\begin{itemize}
\tightlist
\item
  Desarrollar un modelo para la predicción de resultados en las carreras de Fórmula 1.
\item
  Comparar algoritmos de predicción para escoger el óptimo.
\item
  Desarrollar una aplicación para  la interacción con el modelo final.
\end{itemize}

\section{Objetivos técnicos}\label{objetivos-tecnicos}

\begin{itemize}
\tightlist
\item
  Desarrollar un modelo de predicción con el lenguaje Python que gestione toda la complejidad.
\item
  Desarrollar una aplicación en el entorno A DECIDIR.
\item
  Utilizar Git como sistema de control de versiones distribuido junto con
la plataforma GitHub y ZenHub para una gestión más ágil.
\end{itemize}

\section{Objetivos personales}\label{objetivos-personales}

\begin{itemize}
\tightlist
\item
  Profundizar en el manejo de metodologías y herramientas en el campo de la inteligencia artificial que son utilizadas en el mercado laboral.
\end{itemize}


\capitulo{3}{Conceptos teóricos}

Para comprender el marco teórico del desarrollo de este proyecto, es fundamental contar con un conocimiento previo de los conceptos en los que se basa.


\section{Fórmula 1} 
La \href{https://www.formula1.com/}{Fórmula 1} es una competición automovilística que dió inicio en 1950. En la actualidad es considerada la más prestigiosa del mundo y también es la serie deportiva anual con mayor popularidad. Este campeonato está regulado por la la FIA (Federación Internacional del Automóvil) y actualmente está compuesta por 10 equipos con dos coches cada uno. Dicho certámen se lleva a cabo desde marzo hasta noviembre, con un total de 23 carreras disputadas en 20 países distribuidos en cuatro continentes \cite{f1:f1}.

clasificación del mundial de pilotos y equipos--------------------------------------

\section{Modelo predictivo}

Un modelo predictivo es un modelo de datos, basado en estadísticas inferenciales, que se utiliza para predecir resultados. Consiste en un conjunto de herramientas y técnicas estadísticas que sirven para pronosticar y predecir el comportamiento ante un evento. 

Aplicado al campo de la inteligencia artificial, podríamos decir que un modelo de predicción no es más que un modelo de caja negra para hacer predicciones \cite{art:predictmodel}, lo que significa que no se requiere un conocimiento detallado de su funcionamiento interno para utilizarlo. Los modelos predictivos utilizan técnicas basadas en el aprendizaje automático para hacer predicciones basadas en datos históricos o patrones identificados en los datos de entrenamiento. Dichos modelos aprenden automáticamente de los datos de entrenamiento y generan predicciones basadas en su capacidad para identificar patrones ocultos en los datos de entrada

\section{IA - Inteligencia Artificial}

La Inteligencia Artificial o IA se podría definir como la capacidad de las máquinas para tomar decisiones tal como lo hace el ser humano. Para ello se vale de diferentes algoritmos y en el aprendizaje sobre una gran cantidad de datos. Algunos sostienen que la inteligencia radica en las propiedades de los procesos internos de pensamiento y razonamiento, mientras que otros la definen en términos de comportamiento inteligente, una caracterización externa \cite{aimodaproach}.

\section{Aprendizaje automático}
El aprendizaje automático es...

\section{Python}

Python \cite{python} es un lenguaje de programación orientado a objetos, interpretado e interactivo que posee características como la inclusión de módulos, excepciones, tipado dinámico, tipos de datos de alto nivel y clases. Debido a su gran cantidad de librerías y programadores, es muy popular y ampliamente utilizado en diferentes campos de la programación, lo que permite realizar diversos propósitos. Además, se le considera un lenguaje de alto nivel que ofrece una sintaxis clara y sencilla de comprender. \\
Fue creado a principios de la década de 1990 por Guido van Rossum en Stichting Mathematisch Centrum (CWI) en los Países Bajos como sucesor de un idioma llamado ABC \cite{pythonhistory}. \\
Además, este lenguaje de programación es muy popular y ampliamente utilizado en el campo de la inteligencia artificial, ofreciendo una serie de librerías especializadas como pueden ser TensorFlow, Keras, PyTorch, y scikit-learn, que facilitan la implementación de algoritmos de aprendizaje automático y procesamiento de datos. \\
\imagen{most_used_languages}{lenguajes de programación mas usados desde 2010 a 2022 \cite{pythonuse}. Basado en el porcentaje de preguntas en \textit{Stack Overflow}, uno de los portales de preguntas y respuestas sobre programación más importantes de todo el mundo.}{}
\section{Referencias}

Las referencias se incluyen en el texto usando cite \cite{wiki:latex}. Para citar webs, artículos o libros \cite{wiki:latex}.


\section{Imágenes}

Se pueden incluir imágenes con los comandos standard de \LaTeX, pero esta plantilla dispone de comandos propios como por ejemplo el siguiente:

\imagen{escudoInfor}{Autómata para una expresión vacía}{.5}



\section{Listas de items}

Existen tres posibilidades:

\begin{itemize}
	\item primer item.
	\item segundo item.
\end{itemize}

\begin{enumerate}
	\item primer item.
	\item segundo item.
\end{enumerate}

\begin{description}
	\item[Primer item] más información sobre el primer item.
	\item[Segundo item] más información sobre el segundo item.
\end{description}
	
\begin{itemize}
\item 
\end{itemize}

\section{Tablas}

Igualmente se pueden usar los comandos específicos de \LaTeX o bien usar alguno de los comandos de la plantilla.

\tablaSmall{Herramientas y tecnologías utilizadas en cada parte del proyecto}{l c c c c}{herramientasportipodeuso}
{ \multicolumn{1}{l}{Herramientas} & App AngularJS & API REST & BD & Memoria \\}{ 
HTML5 & X & & &\\
CSS3 & X & & &\\
BOOTSTRAP & X & & &\\
JavaScript & X & & &\\
AngularJS & X & & &\\
Bower & X & & &\\
PHP & & X & &\\
Karma + Jasmine & X & & &\\
Slim framework & & X & &\\
Idiorm & & X & &\\
Composer & & X & &\\
JSON & X & X & &\\
PhpStorm & X & X & &\\
MySQL & & & X &\\
PhpMyAdmin & & & X &\\
Git + BitBucket & X & X & X & X\\
Mik\TeX{} & & & & X\\
\TeX{}Maker & & & & X\\
Astah & & & & X\\
Balsamiq Mockups & X & & &\\
VersionOne & X & X & X & X\\
} 

\capitulo{4}{Técnicas y herramientas}

Esta parte de la memoria tiene como objetivo presentar las técnicas metodológicas y las herramientas de desarrollo que se han utilizado para llevar a cabo el proyecto. Si se han estudiado diferentes alternativas de metodologías, herramientas, bibliotecas se puede hacer un resumen de los aspectos más destacados de cada alternativa, incluyendo comparativas entre las distintas opciones y una justificación de las elecciones realizadas. 
No se pretende que este apartado se convierta en un capítulo de un libro dedicado a cada una de las alternativas, sino comentar los aspectos más destacados de cada opción, con un repaso somero a los fundamentos esenciales y referencias bibliográficas para que el lector pueda ampliar su conocimiento sobre el tema.


\section{Herramientas de control de versiones}

\subsection{Github}

\subsection{Github aplicacion escritorio}

\section{Herramientas de gestión de proyectos}

\subsection{ZenHub}

\section{Metodologías}

\subsection{\textit{Sprints}}

\section{Patrones de diseño}

\section{Herramientas de evaluación de código}

\subsection{SonarQube}

\section{Herramientas de documentación}

\subsection{\LaTeX}

\subsection{Overleaf}

\subsection{Draw.io - Microsoft Visio}

\section{Entornos de desarrollo integrado (IDE)}

\subsection{Visual Studio Code}
Usaremos esta interfaz de desarrollo por sus grandes ventajas como puede ser su integración nativa con git, el soporte multiplataforma y multilenguaje, la alta personalización con gran cantidad de extensiones, además de ser gratuito y de código abierto.

\section{Lenguajes de programación}

\subsection{Python}
Se ha escogido python como lenguaje en el desarrollo del modelo porque es uno de los lenguajes más utilizados en el campo de la Inteligencia Artificial. Además de tener una sintaxis clara y simple, existe una amplia comunidad de desarrolladores, y bibliotecas especializadas en IA. Asimismo, es muy flexible y fácil de usar, lo que permite una fácil integración con otros lenguajes.

\subsection{CSharp ¿?}

\subsection{SQL}

\section{Librerías}



\capitulo{5}{Aspectos relevantes del desarrollo del proyecto}

Esta sección tiene como objetivo recopilar las facetas más fascinantes de la evolución del proyecto, desde la exposición del ciclo de vida empleado hasta los detalles más cruciales de las fases de análisis, diseño e implementación.

\section{Elección del tema}

Desde mi infancia he tenido la suerte de poder interactuar con el apasionante mundo del deporte y más en concreto de la Fórmula 1. Puedo recordar claramente haber visto carreras en la televisión con mi familia y maravillarme con la velocidad, la habilidad y la emoción que caracterizan a este deporte de élite. También, esa época coincidió con el éxito de nuestro representante en este deporte, Fernando Alonso. Viví sus dos mejores años, 2005 y 2006. Jamás olvidaré sus luchas con Michael Schumacher, considerado uno de los mejores pilotos de la historia sino el mejor.

Con el tiempo mi terés por la Fórmula 1 se hizo más fuerte. Comencé por investigar a los pilotos más laureados y a los equipos míticos que han dejado una larga huella en la historia de este deporte como pueden ser Ferrari o McLaren. Más adelante continué por interesarme en la evolución de los monoplazas como resultado de los avances tecnológicos y viendo como cada vez se volvían más rápidos, precisos y efectivos.

Es por todo esto que decidí realizar mi Trabajo de Fin de Máster sobre este increíble deporte que es la Fórmula 1. Además, en él es extremadamente importante el uso de modelos predictivos, tanto para predecir el ritmo del coche y el de sus rivales, como para controlar el desgaste de los neumáticos o la potencial pérdida de tiempo en boxes durante una carrera.

\section{Comienzo del proyecto}

Una vez escogido el tema, tocaba planificar el desarrollo del proyecto. Para ello decidí llevar acabo una planificación por sprints. Comenzando por la compresión del problema y la recolección de datos, siguiendo por la creación, entrenamiento y ajuste de los modelos, y acabando por el desarrollo de una aplicación para la interacción con los mismos.

Para llevar a cabo este proyecto decidí escoger GitHub como herramienta de control de versiones, ya que es uno de los requisitos de este proyecto, además de estar ampliamente familiarizado con su tecnología, y Zenhub cómo herramienta de desarrollo ágil, la cual ya he usado con anterioridad. Es aquí dónde encontré mi primer problema en este propósito. Zenhub dispone de un programa de estudiantes que permite utilizar su herramienta de forma gratuita, y así empecé planificando el primer sprint y las primeras tareas. Fue en la preparación del segundo sprint dónde tuve el problema, ese programa gratuito se saturó, por lo que lo inhabilitaron, quedando mi cuenta fuera del mismo. Todo esto hizo que se me complicara la gestión de tareas del proyecto.

\section{Recolección de datos}

Para la recolección de datos inicialmente pasé por contactar con varias páginas web que albergan gran cantidad de información sobre la Fórmula 1. Contacté con 4 de las páginas más relevantes de este deporte: \href{https://www.statsf1.com/}{\textit{statsf1}}, \href{https://www.racing-reference.info/}{\textit{racing-reference}}, \href{https://www.f1-fansite.com/}{\textit{f1-fansite}}, y \href{https://www.motorsportstats.com/}{\textit{motorsportstats}}. Incluso contacté con la página oficial de la \href{https://www.formula1.com/}{\textit{Fórmula 1}}. Algunas de las páginas no me contestaron y otras simplemente me respondieron que no sería posible cederme datos, que no es la política de empresa. 

Tras ver que no fue posible obtener los datos de alguna empresa, tomé la inevitable decisión de recopilarlos yo mismo. Para ello existe de una API llamada \href{https://ergast.com/mrd/}{\textit{Ergast Developer API}}. 

El término \textit{API} (Application Programming Interface o Interfaz de programación de aplicaciones) se refiere a un conjunto de pautas y protocolos que nos permiten la comunicación entre varios componentes de software. Es una interfaz que especifica cómo podemos hacer uso e interactuar con las funcionalidades de un sistema, biblioteca, framework o servicio. 

En el caso de Ergast Developer API es un servicio web experimental que proporciona un registro histórico de datos de carreras de coches para fines no comerciales\cite{eargast:API}. Este servicio nos proporciona datos del deporte de la Fórmula 1, desde el inicio del campeonato en 1950 hasta el día de hoy.

Después de conocer la fuente inicié la recopilación de datos, realizado en varias fases:
\begin{enumerate}
    \item Inicialmente se obtuvieron los datos de todos los pilotos, constructores o equipos y circuitos de la Fórmula 1. Para obtener datos de los circuitos quise obtener la altitud en cuanto a la posición geográfica, ya que afecta notablemente al rendimiento del motor. Para ello me creé una cuenta en la herramienta para desarrolladores de Google, la Google Cloud Platform (GCP). La GCP es una plataforma de servicios de computación en la nube de Google. Se creó para ayudar a las empresas y desarrolladores a crear, implementar y administrar sus aplicaciones y servicios en la nube. Además, nos ofrece una amplia gama de servicios y herramientas. Más concretamente nos proporciona una API llamada \textit{Maps Elevation API}, la cual nos permite a partir de un punto de latitud y longitud, la elevación del terreno entre otros datos. A partir de la información que nos da Ergast (latitud y longitud) he obtenido la altitud geográfica de los circuitos.
 
    \item Más adelante se obtuvieron todos los datos de los resultados de las carreras y las clasificaciones. En este momento nos surge otro problema, sólo obtenemos datos desde el año 2003 de las clasificaciones. Decidí sacar los datos de la página oficial de la Fórmula 1 con técnicas de \textit{webscrapping}. A continuación tuve que crear varios diccionarios para poder asociar a qué circuito y piloto corresponden dichos datos, ya que no cuentan el mismo nombre exacto en la página de la Formúla 1 que en la API de Ergast. Pero al recopilar y hacer el cruce de todos los datos me encontré con que no están todos. Faltan la mayor parte de los datos de clasificación y sólo disponemos de parte de los datos de posición de salida. Por ello usaremos únicamente la información obtenida en la API de Ergast, la cual dispone de los datos de posición de salida para todas las carreras.

    \item En tercer y último lugar lugar quise añadir el dato del tiempo meteorológico de las carreras, que es un factor muy influyente en las carreras. Cuando una carrera es bajo lluvia la influencia del rendimiento del coche baja considerablemente y el desempeño del piloto es de mayor importancia. Es por ello que es un factor muy relevante en las carreras y por lo tanto debe tenerse en cuenta.
    Estos datos no se encuentran en ninguna de las webs de las carreras mencionadas anteriormente y en la API de Ergast tampoco. En un primer lugar consideré a través de la posición geográfica de los circuitos y de la fecha de realización de las carreras, consultar en alguna API de empresas o servicios de meteorología. Tras hacer varias pruebas con la API archive-api.open-meteo.com y la API de wunderground (api.weather.com), no se pudo determinar el tiempo exacto durante la carrera ya que podría haber llovido ese día pero no durante la realización del gran premio. 
    Después de unos días reflexionando sobre cómo abordar este problema, encontré que en wikipedia se muestra cierta información sobre la temperatura del asfalto y sobre si llovió o no. Es por esto que mediante técnicas de \textit{webscrapping} y la librería \textit{BeautifulSoup}, creando un diccionario que detecta si hay palabras que indiquen que ha llovido, se ha establecido si llovió (representado con el valor \textit{wet}) o no (\textit{dry}) durante el gran premio.
\end{enumerate}

\section{Preparación y limpieza de datos}

Tras la recopilación de los datos se procedió a prepararlos y limpiarlos para el modelo. 

\subsection{Fase 1, estados}

En este momento decidí comenzar con el estado de finalización de las carreras. En la API de Ergast además de la posición de finalización de una carrera tenemos el estado en que se terminó esta, a continuación mostramos los estados más comunes en orden descendentes en la siguiente tabla \ref{tabla:estadosfinalcarrerasf1}.

\tablaSmall{Estados de finalización de un piloto en una carrera.}{l c}{estadosfinalcarrerasf1}
{ \multicolumn{1}{l}{Estados} & Representación del estado \\}{ 
Finished & Terminó la carrera sin problemas \\
+1 Lap & Terminó la carrera con una vuelta de retraso \\
Engine & Problemas en el motor del vehículo \\
Accident & Estuvo involucrado en un accidente \\
Collision & Estuvo involucrado en una colisión \\
Spun off & Se salió de pista durante la carrera \\
Gearbox & Problemas en la caja de cambios \\
Did not qualify & No logró calificar para la carrera \\
Suspension & Problemas en la suspensión del vehículo \\
Electrical & Problemas eléctricos en el vehículo \\
Transmission & Problemas en la transmisión del vehículo \\
Brakes & Problemas en los frenos del vehículo \\
Clutch & Problemas en el embrague del vehículo \\
}

Había demasiados estados finales, por lo que los reduje a tres estados para simplificarlo. Podemos verlos en la tabla \ref{tabla:estadossimplesfinalcarrerasf1}. 

\tablaSmall{Estados de finalización simplificados.}{l c}{estadossimplesfinalcarrerasf1}
{ \multicolumn{1}{l}{Estados} & Representación del estado \\}{ 
Finished & Terminó la carrera sin problemas \\
Driver mistake & No terminó por un error de pilotaje \\
Mechanical failure & No terminó por problemas mecánicos del vehículo \\
Engine failure & No terminó por problemas en el motor del vehículo \\
}

\subsection{Datos del piloto, circuitos y constructores}
Más adelante crucé el \textit{dataset} con los datos del piloto, de los circuitos y de los equipos, los cuales había obtenido con anterioridad. Aquí se contemplan datos como por ejemplo la nacionalidad del piloto, el país donde se ubica el circuito o la nacionalidad del equipo. 
A la hora de obtener la nacionalidad y el país del equipo o circuito, observé que para el modelo sería interesante que ese dato fuera igual en los tres casos. Es por ello que modifiqué la nacionalidad de los pilotos para que aparezca en forma país y no de nacionalidad. 
Por otra parte, como amante del deporte que soy, tuve en cuenta que por temas de patrocinios muchos equipos cambian constantemente de nombre durante su historia, y es muy imperante que se tenga esto en cuenta para que los datos de un mismo equipo no se traten de forma separada. Con el fin de unificar los nombres del mismo equipo, busqué información sobre la historia de todos los constructores, la cual podemos ver la siguiente imagen \ref{fig:constructor_history_min}, para modificar este dato.

\imagen{constructor_history_min}{Extracto de los cambios de nombre en la historia de algunas de las escuderías\cite{reddit:f1teamhistory}.}{1}

Además, utilicé la fecha de realización del gran premio y la fecha de nacimiento del piloto para calcular la edad del piloto durante la carrera. Esto es importante, ya que con el tiempo los pilotos están más experimentados, por lo cual son más rápidos y cometen menos errores como podemos ver en la figura \ref{fig:errorespilotoedad}.

\imagen{errorespilotoedad}{Promedio de errores de los pilotos según la edad}{.8}

Por último se obtuvieron las puntuaciones de cada año del campeando tanto en constructores como en pilotos. Estos datos se pueden obtener directamente de la API de Eargast. Pero cómo la puntuación en función del resultado en carrera ha sido modificado con el tiempo, se va a proceder a adjudicar los puntos en función de la posición final de cada carrera con el sistema de puntuación actual. En este sistema sólo puntúan los 10 primeros pilotos, tal como podemos ver en la tabla \ref{tabla:puntuacioncarrerasf1}.
    \tablaSmall{Puntuación de cada piloto en función de la posición final de carrera.}{l c}{puntuacioncarrerasf1}
    { \multicolumn{1}{l}{Posición} & Puntuación \\}{ 
        1° & 25\\
        2° & 18\\
        3° & 15\\
        4° & 12\\
        5° & 10\\
        6° & 8\\
        7° & 6\\
        8° & 4\\
        9° & 2\\
        10° & 1\\
    } Esto se utilizará más adelante para comprobar el error de los modelos.

\subsection{Fase 3, Ganadores, \textit{polemans}, fiabilidad de coches y consistencia de pilotos}
Con los datos de las posiciones de salida obtendremos los \textit{polemans}. La \textit{pole} es para el piloto que consigue hacer la vuelta más rápida en clasificación y por ello sale en primer lugar en la carrera. El \textit{poleman} es el piloto que consigue la \textit{pole}. Es interesante conocer este dato porque en muchos circuitos en los cuales es muy difícil adelantar y salir por delante es de gran importancia como vemos en el gráfico de la figura \ref{fig:winnerpole}, el 40\% de las victorias es saliendo desde la primera posición.

\imagen{winnerpole}{Porcentaje de victorias cuando el piloto sale en la primera posición.}{.6}

Y si comparamos las victorias saliendo desde la tercera posición o mejor el dato es aún más relevante (figura \ref{fig:winneron3grid}), un 80\%.

\imagen{winneron3grid}{Porcentaje de victorias cuando el piloto sale en las tres primeras posiciones.}{.8}

Además, vamos a obtener el ganador del gran premio a través de las posiciones de salida, para más adelante poder predecir este dato.

De igual modo es importante conocer la fiabilidad de los coches y la consistencia de los pilotos. En este caso usaremos la información de finalización de carrera (tabla \ref{tabla:estadossimplesfinalcarrerasf1}), en función del total de carreras. Para la consistencia del piloto se usará el estado \textit{driver mistake} y para la fiabilidad de los coches los estados \textit{mechanical failure} y \textit{engine failure}.

\subsection{Limpieza de datos}

Más adelante se procedió con la limpieza de datos, ya que debemos comprobar si hay algún dato que sea nulo, ya que los valores nulos afectarán negativamente a la eficiencia de los modelos. 
Al comprobar los datos inexistentes, se eliminará la siguiente información: 

\begin{itemize}
	\item Tiempos de carrera: no contamos ni con el 50\% de los tiempos.
	\item Vuelta rápida de piloto en carrera: al igual que con los tiempos de carrera falta más de la mitad de los datos, y por ello la velocidad media de esa vuelta también será eliminada.
    \item Código de piloto y número de piloto: estos números no aportan información y además no hay prácticamente registros de ello.
\end{itemize} 

Datos descargados que se eliminaran porque no aportan ninguna información sobre las carreras: 
    \begin{itemize}
        \item Urls tanto de pilotos como de circuitos, carreras, o constructores.
        \item Nombres y/o apellidos de pilotos y constructores, ya que contamos con el id o alias de cada uno de ellos.
        \item Nacionalidad de piloto o equipo, y nombre de la localidad del circuito. Como ya hemos convertido la nacionalidad a nombre del país, ya no es necesario conocer dicha información.
    \end{itemize}


\section{Selección de características y codificación de los datos}

En este punto se deben escoger las características que son más importantes para el entrenamiento del modelo. Para lograr esto, contrastaremos el uso de un algoritmo de selección de características frente a la selección manual de los rasgos que, en mi opinión, son más cruciales según mis muchos años de experiencia siguiendo este deporte. 

\subsection{Codificación de los datos}

Es necesario tener los datos codificados para realizar la selección automática de características, ya que los modelos trabajan con datos numéricos.

En el aprendizaje automático existen dos grandes codificadores de características en función del tipo de variable. 
\begin{enumerate} 
    \item Para variables categóricas con diferentes categorías se suele utilizar la codificación \textit{One-Hot}, en la cual se crea una columna binaria para cada categoría.
    \item En variables con un orden jerárquico se usan codificaciones ordinales, las cuales asignan a cada valor diferente de la variable un valor numérico normalmente ascendente o descendente.
\end{enumerate}

Para nuestros datos se han escogido las siguientes codificaciones para cada uno de las variables:

\begin{itemize}
    \item Año, posición final en carrera, posición de salida, ronda o serie, vueltas terminadas en carrera, puntos obtenidos tras la carrera, latitud o longitud de la posición geográfica del circuito, altitud del circuito respecto del nivel del mar, edad del piloto, consistencia del piloto y fiabilidad del coche: estos datos ya son numéricos así que no es necesario codificarlos.
    \item Fecha del gran premio: para este dato hemos pasado la fecha a segundos con la referencia de la fecha mas antigua.
    \item Id de circuito, piloto y equipo: para esta variable se ha usado la codificación ordinal para asignar a cada piloto, circuito y equipo un número diferente de cada variable.
    \item Estados de finalización de carrera (estado final y estado simple final): para los estados se ha escogido también una codificación ordinal, asignando un número a cada variable.
    \item Clima: como tenemos dos tipos de climas se ha decidido codificarlos utilizar el método \textit{One-Hot}, por lo que pasamos a tener dos variables en vez de una. Cada variable indica si llovió o no durante la carrera.
    \item Fecha de nacimiento del piloto: para este dato hemos utilizado el mismo criterio que con la fecha del gran premio, hemos pasado ese día a segundos con la referencia de la fecha mas antigua de nacimiento.
    \item País de procedencia de piloto, equipo y circuito: se ha considerado que este dato debía ser codificado con \textit{One-Hot} debido a que el algoritmo debe poder detectar cuando un piloto o equipo corre en el gran premio de su país de procedencia de una mejor forma.
    \item Ganador del gran premio: no se ha codificado ya que ya es un dato numérico que representa un 1 si el piloto queda en primer lugar tras la carrera o un 0 si no.
    \item \textit{Poleman} del gran premio: de igual modo que el ganador del gran premio no se ha necesitado codificar.
\end{itemize}

\subsection{Selección de características manual}

En primer lugar hice la selección manual. Gracias a este enfoque logro tener un mayor control sobre qué características se incluyen o excluyen en el modelo final, lo que me brinda la oportunidad de aplicar mi conocimiento y experiencia en este campo. 
Estas son las características elegidas:

\begin{itemize}
    \item Año: el año es importante ya que tenemos que conocer este dato para distinguir entre carreras de un año y otro.
    \item Id de circuito, piloto y equipo: valores necesarios para diferenciar entre pilotos y equipos en cada circuito.
    \item Posición de salida del piloto: esta variable es de suma importancia en circuitos de complejidad de adelantamiento.
    \item Posición final del piloto en carrera: este dato es una de las variables objetivo del proyecto.
    \item Clima en carrera: uno de los valores más importantes en mi opinión, en mojado los rendimiento de los coches no son demasiado importantes y cobra mayor relevancia la destreza del piloto.
    \item País de procedencia de piloto, equipo y circuito: los pilotos de carreras suelen tener mayor motivación cuando corren en casa, al igual que cuando corren en el país de procedencia del equipo.
    \item Altitud sobre el nivel del mar donde se encuentra el circuito: los motores de los coches de Fórmula 1 se ven afectados por este dato, esto es por la cantidad de oxigeno en el aire. A mayor altitud menor concentración de oxígeno y mayor exigencia para los motores, ya que son motores de combustión y el oxígeno es de vital importancia en la explosión del combustible.
    \item Edad del piloto durante el gran premio: la edad es importante, ya que cuando un piloto es joven es menos experimentado y suele arriesgar más y por tanto comete más errores. Pero también cuando los pilotos son muy mayores suelen perder reflejos y algunos pierden la motivación para correr. Esto lo pudimos ver anteriormente en la figura \ref{fig:errorespilotoedad}.
    \item Ganador de la carrera: variable objetivo del proyecto.
    \item Poleman de la carrera: variable objetivo del proyecto.
    \item Consistencia del piloto: esta variable nos indica la consistencia del piloto a la hora de cometer errores, esto es importante porque cuantos más errores cometen más fácil es que no acaben la carrera.
    \item Fiabilidad del coche: es muy importante conocer este dato, en caso de un coche con poca fiabilidad es posible que el coche sufra un fallo en carrera y abandone. Podemos recordar la época de Fernando Alonso en McLaren con el motor honda entre los años 2015 y 2018 con la fiabilidad de los coches ese año, ilustrado en la figura \ref{fig:fiabilidadcar2015-2018}. En dos de cada 10 carreras fallaban los dos coches abandonando por fiabilidad. Podemos decir sin lugar a dudas que fue el peor coche en cuanto a fiabilidad de esa época.
    \imagen{fiabilidadcar2015-2018}{Fiabilidad de los coches entre 2015 y 2018}{1}
\end{itemize} 

\subsection{Selección de características automática}

Para esta tarea contemplamos dos métodos muy efectivos, la eliminación recursiva o \textit{Recursive Feature Elimination} y la eliminación hacia atrás o \textit{Backward Elimination}.

La técnica de eliminación hacia atrás comienza con todas las características y elimina una a una en cada iteración. Entonces, en cada iteración de esta técnica se entrena y evalúa un modelo con las variables disponibles. La característica menos significativa es la eliminada, tomando como criterio el peso de la misma en esa iteración. Esto se repite hasta llegar al criterio de parada.

Elgoritmo RFE o \textit{Recursive Feature Elimination}, realmente es un tipo de eliminación hacia atrás, pero tiene la ventaja de ser compatible con varios algoritmos de aprendizaje automático, lo que lo convierte en una herramienta flexible y adaptable. Se puede combinar con algoritmos de regresión, clasificación u otras estrategias de modelado, brindándonos la flexibilidad de manejar varios problemas y conjuntos de datos. Además, proporciona una medida de importancia o relevancia para cada característica en función de cómo contribuye al modelo final. Dado que nos permite identificar los rasgos que tienen mayor influencia en la toma de decisiones del modelo, será particularmente útil para nuestro propósito.

Este algoritmo se ha combinado con varios modelos para hacer una selección más óptima. Estos modelos son: \textit{Decision Tree Regressor} y \textit{Linear Regression}. Los dos son modelos de regresión ya que queremos una elección lineal de características. Por un lado los árboles de decisión son modelos muy versátiles que nos van a permitir capturar las relaciones no lineales y podremos detectar interacciones entre las diferentes características. Y por otra parte el algoritmo de regresión lineal tiene la ventaja de ser más interpretable y computacionalmente más eficiente.

Tras la ejecución del algoritmo nos han quedado las siguientes características:

race\_final\_position...
\begin{enumerate}
    \item Fiabilidad del coche
    \item Consistencia del piloto
    \item Puntos del piloto en carrera
    \item País del piloto, equipo y circuito
    \item Estado final del piloto en carrera
    \item Vueltas finalizadas
    \item Clima mojado y clima seco
    \item Ganador
    \item Posición de salida
    \item Estado final
    \item Altitud del circuito sobre el nivel del mar
    \item Año
    \item Posición geográfica en latitud y longitud
    \item Serie
\end{enumerate}

race\_winner...
\begin{enumerate}
    \item Consistencia del piloto
    \item Piloto en \textit{pole}
    \item Fiabilidad del coche
    \item Puntos del piloto en carrera
    \item Estado final
    \item Posición geográfica en latitud y longitud
    \item Edad del piloto durante el gran premio
    \item País del piloto, equipo y circuito
    \item Altitud del circuito sobre el nivel del mar
    \item Clima seco
    \item Fecha de la carrera
    \item Clima mojado
    \item Fecha de nacimiento del piloto
    \item Posición final del piloto en carrera
    \item Posición de salida
\end{enumerate}

qualy\_pole...
 \begin{enumerate}
    \item Ganador
    \item Consistencia del piloto
    \item Fiabilidad del coche
    \item País del piloto, equipo y circuito
    \item Posición de salida
    \item Estado final
    \item Posición geográfica en latitud
    \item Puntos del piloto en carrera
    \item Posición final del piloto en carrera
    \item Posición geográfica en longitud
    \item Clima seco
    \item Vueltas finalizadas
    \item Clima mojado
    \item Edad del piloto durante el gran premio
    \item Altitud del circuito sobre el nivel del mar
\end{enumerate}

\capitulo{6}{Trabajos relacionados}

Este apartado sería parecido a un estado del arte de una tesis o tesina. En un trabajo final grado no parece obligada su presencia, aunque se puede dejar a juicio del tutor el incluir un pequeño resumen comentado de los trabajos y proyectos ya realizados en el campo del proyecto en curso. 

\section{A machine learning approach to predict the winner of the next F1 Grand Prix - Veronica Nigro}
Este trabajo es un proyecto desarrollado por Veronica Nigro. Este proyecto se encuentra documentado en un \href{https://towardsdatascience.com/formula-1-race-predictor-5d4bfae887da}{artículo} de \href{https://towardsdatascience.com/}{Towards Data Science}, una plataforma en línea dedicada a compartir información y recursos sobre ciencia de datos, inteligencia artificial y aprendizaje automático. Este proyecto ha construido un algoritmo para predecir el ganador en las 21 carreras de Fórmula 1 de la temporada 2019 \cite{towardsdatascience:f1veronigro}. Podemos encontrar el repositorio que alberga el proyecto en el siguiente \href{https://github.com/veronicanigro/Formula_1/blob/master/README.md}{enlace} o en el propio artículo. 


\section{Formula 1 Race Prediction}
Formula 1 Race Prediction es un proyecto de aprendizaje automático utilizando el módulo Scikit-learn en Python para predecir el resultado de las carreras de Fórmula 1. El objetivo de este predictor es predecir si el piloto gana (primera posición), entra en los puntos (entre el segundo y el décimo piloto) o se queda fuera de los puntos (undécimo o peor), además cuenta con una interfaz gráfica que proporciona esta información (podemos verlo en la figura \ref{fig:trabajo_relacionado_2}). El trabajo se encuentra alojado en el repositorio \href{https://github.com/Sloopy3333/Formula-1-Prediction}{Formula 1 Race Prediction}.

\imagen{trabajo_relacionado_2}{Interfaz de Formula 1 Race Prediction.}{.8}

\section{Ventajas y debilidades respecto a la competencia}

\begin{table}[h!]
\centering
\begin{adjustbox}{width=\textwidth}
\begin{tabular}{lccc}
\toprule
Características                 & AppFran     & F1 Grand Prix 
Veronica Nigro & Formula 1 Race Prediction   \\
\midrule
Predicción del ganador               & \cellcolor{green!25} {$\checkmark$} & \cellcolor{green!25} {$\checkmark$}  & \cellcolor{green!25} {$\checkmark$}  \\
Interfaz de usuario             & \cellcolor{green!25} {$\checkmark$} & \cellcolor{red!25} {$\times$}  & \cellcolor{green!25} {$\checkmark$}  \\
Gratuito                        & \cellcolor{green!25} {$\checkmark$} & \cellcolor{green!25} {$\checkmark$}  & \cellcolor{green!25} {$\checkmark$}  \\
Predicción resultado carrera                        & \cellcolor{green!25} {$\checkmark$} & \cellcolor{red!25} {$\times$}  & \cellcolor{red!25} {$\times$}  \\
Predicción del ganador de la pole                        &
\cellcolor{green!25} {$\checkmark$} & \cellcolor{red!25} {$\times$}  & \cellcolor{red!25} {$\times$}  \\
\bottomrule
\end{tabular}
\end{adjustbox}
\caption{Comparativa de características de los proyectos.}
\label{comparativa-proyectos}
\end{table}

Las principales fortalezas del proyecto son:

\begin{itemize}
\tightlist
\item
  \textbf{Funcionalidades}: Contamos con diferentes tipos de predicciones que permiten al usuario final elegir entre las diferentes opciones. 
\item
  \textbf{Interfaz de usuario}: la interfaz de usuario de la aplicación es muy intuitiva, sencilla y con una curva de aprendizaje rápida. Además, es accesible a todo el mundo, incluso sin conocimiento previo del tema.
\item
  \textbf{Aplicación gratuita}.
\end{itemize}

Las principales debilidades son:

\begin{itemize}
\tightlist
\item
  Cuenta con una menor probabilidad de acierto en el caso de la predicción del resultado total de la carrera.
\item
  Menor dimensionalidad en los datos.
\end{itemize}

\capitulo{7}{Conclusiones y Líneas de trabajo futuras}

\section{Conclusiones}

Procedo a mostrar las conclusiones que se derivan del desarrollo del presente proyecto:

\begin{itemize}
\tightlist
\item
  El objetivo general del proyecto se ha cumplido con éxito. Se ha desarrollado un modelo de predicción de resultados de Fórmula 1, el cual puede utilizarse de una manera muy simple gracias a la interfaz de usuario por cualquier persona.
\item
  También se ha completado el ajuste de los modelos con éxito consiguiendo mejoras notables en alguno de ellos.
\item
  Por otra parte he comprobado que aunque la selección manual de características puede ser un buen enfoque, la selección mediante algoritmos como la eliminación recursiva de características son más eficientes computacionalmente en la mayor parte de las ocasiones.
\item
  Gracias a la realización del proyecto he profundizado en el conocimiento sobre inteligencia artificial, modelos predictivos y machine learning.
\item
  Este proyecto ha abarcado una buena parte de los conocimientos que he obtenido durante mi estancia en la universidad. Pero no sólo he hecho uso de ellos, si no que también he estudiado y adquirido nuevos conocimientos requeridos para la realización del proyecto. Entre ellos se encuentran: Webscrapping, Scikit-learn, \LaTeX, pyqt, etc.
\item
  He aprendido a hacer búsquedas bibliográficas rápidas y efectivas como resultado de los requerimientos de investigación del proyecto.
\item
  El período de desarrollo de este proyecto ha supuesto el uso de numerosas tecnologías y herramientas. Todas ellas han contribuido a elevar su calidad. Sin embargo, algunas de ellas han requerido una sobrecarga de trabajo considerable. A pesar de esto, la información aprendida será muy útil para futuros emprendimientos.

\end{itemize}

\section{Líneas de trabajo futuras}

A continuación, discutiremos cada extensión o mejora potencial que sería aplicable al trabajo actual.

\subsection{Aumentar la dimensionalidad de los datos}
Un punto de mejora en este proyecto sería lograr mayor cantidad de datos de clasificación, recopilar datos de los pit stops o de las vueltas rápidas, adelantamientos, etc.

\subsection{Añadir funciones a la aplicación}
Se podría añadir un apartado en la aplicación para añadir datos al conjunto y que se vuelvan a entrenar los modelos automáticamente, y si los datos mejoran, adoptar esos nuevos modelos.

\subsection{Interpretación de resultados}
Podrían desarrollarse métodos para comprender y justificar las opciones del modelo a medida que interpreta las decisiones. El usuario final puede beneficiarse de esto al sentirse más seguros de las predicciones y al poder comprender las variables que afectan a los resultados.

\subsection{Aplicación en dominios particulares}
Una buena aplicación de este proyecto podría ser su uso en dominios particulares como pueden ser análisis deportivo, simulaciones o videojuegos.



\bibliographystyle{plain}
\bibliography{bibliografia}

\end{document}
